
% Using Tera with Tex has a few gotchas.
% Tera is sensitive to the {{ , {% , and {# sequence;
% because of that, these need to be escaped:
%  - \command{{{ tera_var }}} can be escaped as \command{\I {{ tera_var }}
%    where \I is an alias for \ignorespaces when whitespace IS NOT significant
%  - \command{{{ tera_var }}} can be escaped as \command{\V{}{{ tera_var }}
%    where \V is an alias for \vphantom when whitespace IS significant
%  - {% at the end of a line can be escaped as { % or {\I %
%  - Where \I or \V doesn't work, one needs to use Tera raw blocks
%
% The content of all variables is by default escaped with a TeX
% escaping function similar to how special characters are escaped in HTML.
% The escaping function doesn't escape spaces.
% To preserve spaces, use {{ variable | pre | safe }}, which replaces spaces
% with the non-breaking space TeX entity '~'.


\documentclass[{{ output.fontsize | default(value="12") }}pt,a5paper,openany]{memoir}

\setheadfoot{0pt}{0pt}
\setlrmarginsandblock{1cm}{1cm}{*}
\setulmarginsandblock{.75cm}{1cm}{*}
\checkandfixthelayout
\raggedbottom

\usepackage{fontspec}
\usepackage{titlesec}
\usepackage{xcolor}
\usepackage{enumitem}
\usepackage[defaultlines=5,all]{nowidow}
\usepackage[pdfusetitle,hidelinks]{hyperref}

%% Aliases for \easier escaping
\newcommand\I[0]{%
  \ignorespaces
}
\newcommand\V[0]{%
  \vphantom
}

%% Page style
\makepagestyle{songs}
\makeoddfoot{songs}{}{}{\thepage}
\makeevenfoot{songs}{\thepage}{}{}
% Patch cleardoublepage to not get blank pages after title & contents pages:
\renewcommand\cleardoublepage{\clearpage}
% Suppress page numbers in the toc/frontmatter:
\aliaspagestyle{chapter}{empty}

%% Fonts
\setmainfont[Ligatures=TeX]{
  {{ output.font | default(value="DejaVu Sans") }}
}

%% Spacings
\setlength{\parindent}{0pt}
\setlength{\tabcolsep}{0pt}
\setlength{\parskip}{0pt}

%% ToC style
% Hide the title of the ToC:
\renewcommand\tocheadstart{}
\renewcommand\printtoctitle[1]{}
\renewcommand\aftertoctitle{}
% Hide section numbers in the ToC:
\renewcommand\numberline[1]{}
\renewcommand\cftdotsep{1}

%% Hyperlinks setup
\hypersetup{
  bookmarks=true,
  pdfcreator={{{ program.name }} v. {{ program.version }} - {{ program.homepage }}},
}

%% Song title and subtitle formats
\titleformat{\section}
  {\centering\large\bfseries}{}{0pt}{\underline}
\titlespacing*{\section}
  {0pt}{0pt}{.25ex}
\newcommand\songtitle[1]{%
  % This is a trick to only layout a song on the current page
  % if it fits, otherwise a pagebreak is inserted
  \FloatBlock
  \vfil
  \pagebreak[2]
  \vfilneg
  \section{#1}
}
\newcommand\subtitle[1]{%
  \emph{#1}
}

%% Chord layout command
\makeatletter
\newcommand{\chord}{\@ifstar{\@chordalt}{\@chord}}
\newcommand{\@chord}[2]{%
  \begin{tabular}[b]{l}%
    \textbf{\textcolor{red}{#1}}\\%  % The chord
    #2\mbox{}\end{tabular}%          % The lyrics
}
\newcommand{\@chordalt}[3]{%
  \begin{tabular}[b]{l}%
    \textbf{\textcolor{red}{#1}}\\%  % The chord
    \textbf{\textcolor{blue}{#2}}\\% % The alt chord
    #3\mbox{}\end{tabular}%          % The lyrics
}
\makeatother

%% Time layout command
\newcommand\timesign[2]{%
  \begin{large}\( {#1} / {#2} \)\end{large}
}


% Metadata
\title{\I {{book.title}}}

% Document
\begin{document}

%% Title page
\frontmatter
\begin{titlingpage}
  \begin{vplace}[0.5]
    \begin{center}
      \Huge{\textbf{\I {{ book.title }}}} \\
      \vspace{0.5cm}
      \LARGE{\I {{ book.subtitle | default(value="") }}} \\
      \vspace{10cm}
      \small{\I {{ book.title_note | default(value="") }}}
    \end{center}
  \end{vplace}
\end{titlingpage}

%% Contents page
\tableofcontents*

%% Songs
\mainmatter
\pagestyle{songs}

  %% song {{ loop.index }}
  \songtitle{\I {{ song.title }}}

  
    \begin{center}
    
      \subtitle{\I {{ subtitle }}}\\
    
    \end{center}
  
    \vphantom{}
  

  \begin{description}[leftmargin=2.5em,style=nextline,itemsep=1ex]
    

      
        \item[\emph{\I {{ item.label }}}{{ item.label }}]
        
          
            
              \chord{\V{}{{ span.chord | pre | safe }}}{\V{}{{ span.lyrics | pre | safe }}}
            
              \chord*{\V{}{{ span.chord | pre | safe }}}{\V{}{{ span.chord_alt | pre | safe }}}{#
                #}{\V{}{{ span.lyrics | pre | safe }}}
            
              {{ span.lyrics | pre | safe }}
            
          
          \\
        

      
        \item[] \begin{itemize}[noitemsep]
          
            \item {{ item }}
          
        \end{itemize}

      
        \item[] \timesign{\I {{ item.time[0] }}}{\I {{ item.time[1] }}}

      
        \item[] \vphantom{}\hrule

      
        \item[] \begin{verbatim}{{ item.text | safe }}\end{verbatim}

      
    
      {# No song.content, we need to insert and empty item,
         otherwise LaTeX fails on an empty description env. #}
      \item
    
  \end{description}


\backmatter
\pagestyle{empty}


\pagebreak
\begin{center}
  {{ book.backmatter }}
\end{center}



\pagebreak
\begin{tiny}
  \begin{verbatim}{{ __tera_context | base64 }}\end{verbatim}
\end{tiny}


\end{document}
